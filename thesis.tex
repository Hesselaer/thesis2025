\documentclass[12pt]{article}

\usepackage[left=2.5cm,right=2.5cm,top=2.5cm,bottom=2.5cm]{geometry}
\usepackage{graphicx}
\usepackage[dvipsnames]{xcolor}
\usepackage{tikz}
\input{tikz/interfacepoint.tikzstyles}
\usepackage{tikz-cd}
\usepackage{tikz-qtree}
\usetikzlibrary{shapes,backgrounds,calc,arrows}
\usepackage{microtype}

\usepackage[setpagesize=false, colorlinks=true,urlcolor=red,pdftitle={Simplicial Coalgebras for Concurrent Regular Languages},pdfauthor={Hessel Sieburgh}]{hyperref}
\frenchspacing
\setlength\parindent{0pt}

\usepackage{amsmath}
\usepackage[utf8]{inputenc}
\usepackage{amsfonts}
\usepackage{amssymb}
\usepackage{amsthm}

\newtheorem{definition}{Definition}[section]
\newtheorem{theorem}{Theorem}[section]
\newtheorem{lemma}{Lemma}[section]
\newtheorem{example}{Example}[section]

\usepackage{bbm}
\usepackage{csquotes}
\usepackage{mathtools}
\usepackage{parskip}
\usepackage{mathtools}
\usepackage{mathdots}
%\usepackage[hidelinks]{hyperref}
\newcommand{\defeq}{\vcentcolon=}
\newcommand{\eqdef}{=\vcentcolon}

\DeclareMathOperator{\Ima}{Im}
\DeclareMathOperator{\sech}{sech}
\renewcommand{\P}{\mathcal{P}}
\newcommand{\R}{\mathbb{R}}
\newcommand{\C}{\mathbb{C}}
\newcommand{\N}{\mathbb{N}}
\newcommand{\E}[1]{\mathbb{E}\{#1\}}
\newcommand{\Q}{\mathbb Q}
\newcommand{\Z}{\mathbb{Z}}
\newcommand{\1}{\mathbbm{1}}
\newcommand{\ua}{\nearrow}
\newcommand{\da}{\searrow}
\newcommand{\eps}{\varepsilon}
\newcommand{\dx}{\mathrm{d}x}
\newcommand{\B}{\mathcal{B}}
\newcommand{\F}{\mathcal{F}}
\newcommand{\T}{\mathbb{T}}
\newcommand{\V}{\mathcal{V}}
\newcommand{\U}{\mathcal{U}}
\newcommand{\id}{\text{id}}
\newcommand{\I}{\mathcal{I}}
\newcommand{\A}{\mathcal{A}}
\newcommand{\J}{\mathcal{J}}
\newcommand{\G}{\mathcal{G}}
\renewcommand{\L}{\mathcal{L}}
\newcommand{\specepsilon}{\mathcal{E}}
\newcommand{\M}{\mathcal{M}}
\newcommand{\II}{\textbf{II}}
\newcommand{\bigo}{\mathcal{O}}
\renewcommand{\H}{\mathcal{H}}
\newcommand{\finP}{\mathcal{P}_{\omega}}

\begin{document}
\thispagestyle{empty}

\includegraphics{logoleiden}

\vspace{-2.5cm}\hfill \begin{huge}\textbf{Opleiding Informatica}\end{huge}

\vspace{5cm}
\begin{Large}
\hfill Simplicial Coalgebras

\vspace*{3mm}

\hfill for Concurrent Regular Languages

\vspace*{14mm}

\hfill Hessel Sieburgh
\end{Large}

\vspace*{6.0cm}

\begin{large}

Supervisors:\\
Henning Basold \& 
Marton Hablicsek


\vspace*{1.8cm}
BACHELOR THESIS

\vspace*{5mm}
Leiden Institute of Advanced Computer Science (LIACS)\\
\href{www.liacs.leidenuniv.nl}{\underline{\texttt{www.liacs.leidenuniv.nl}}}\hfill 01/07/2025
\end{large}

\newpage
\begin{abstract}
\noindent
This thesis introduces a construction of automata for concurrent languages. This is done by defining ranked hypergraphs, hypergraphs with interfaces that can be composed associatively. A simplicial set over these graphs is defined and we define F-coalgebras which give a nondeterministic transition model over the cells. The union of all paths of a tree resulting from a certain coalgebra gives a language of traces that support concurrent and sequential composition through the found operations on ranked hypergraphs. 
\end{abstract}

\bigskip

\thispagestyle{empty}
\tableofcontents
\thispagestyle{empty}

\clearpage
\setcounter{page}{1}

\section{Introduction} \label{introduction}

In this section we give an introduction to the problem addressed in this thesis.

% \begin{figure}
%     \centering
%     \begin{tikzpicture}
% 		\node [style={gra_point}, label={above:1}] (0) at (-1, 1.5) {};
% 		\node [style={gra_point}, label={above:2}] (1) at (-1, 0) {};
% 		\node [style={gra_point}, label={above:4}] (3) at (1, 1.5) {};
% 		\node [style={int_point}] (5) at (-3, 1.5) {};
% 		\node [style={int_point}] (6) at (-3, 0) {};
% 		\node [style={int_point}] (7) at (-3, -1.75) {};
% 		\node [style={int_point}] (8) at (3.75, 1.25) {};
% 		\node [style={int_point}] (9) at (3.75, -1.25) {};
% 		\node [style={gra_point}, label={above:6}] (10) at (1, -0.75) {};
% 		\node [style={gra_point}, label={above:5}] (11) at (1, 0.5) {};
% 		\node [label={above:a}] (12) at (0, 0) {};
% 		\node [label={above:b}] (13) at (0, 1.5) {};
% 		\node [style={gra_point}, label={above:3}] (14) at (0, -1.75) {};
% 		\draw [style={int_edge}] (5) to (0);
% 		\draw (1) to (12.center);
% 		\draw [style={int_edge}] (7) to (14);
% 		\draw [style={int_edge}] (6) to (1);
% 		\draw [style={int_edge}] (9) to (14);
% 		\draw [style={int_edge}] (9) to (10);
% 		\draw [style={int_edge}] (8) to (3);
% 		\draw [style={gra_edge}] (0) to (3);
% 		\draw [style={gra_edge}] (12.center) to (11);
% 		\draw [style={gra_edge}] (12.center) to (10);
%     \end{tikzpicture}
%     \caption{A Directed Acyclic (ranked) Hypergraph}
%     \label{fig:intro-DAH}
% \end{figure}


\subsection{The problem}

\subsection{Earlier research}

\subsection{Thesis overview}

\newpage
\section{Background}
\subsection{Simplicial sets}
\begin{definition}
    A \emph{simplicial set} is a presheaf on the simplex category \( \Delta \), which means a simplicial set is a functor
    \[
    X \colon \Delta^{\mathrm{op}} \to \mathbf{Set}.
    \]
    
    The category \( \Delta \) has as objects
    \[
    [n] = \{0 < 1 < \dots < n\}
    \]
    and as morphisms the order-preserving functions between them.
    
    Thus, a simplicial set \( X \) assigns to each \( [n] \in \Delta \) a set \( X_n = X([n]) \) of \emph{n-simplices}, and to each morphism \( \theta \colon [m] \to [n] \), a function \( X(\theta) \colon X_n \to X_m \).
    
    In particular, arise from:
    \begin{itemize}
      \item The \emph{face maps} \( d_i \colon X_n \to X_{n-1} \) are given by \( X(\delta^i) \), where \( \delta^i \colon [n-1] \to [n] \) skips \( i \).
      \item The \emph{degeneracy maps} \( s_i \colon X_n \to X_{n+1} \) are given by \( X(\sigma^i) \), where \( \sigma^i \colon [n+1] \to [n] \) repeats \( i \).
    \end{itemize}
    
    These satisfy the \emph{simplicial identities}:
    \[
    \begin{aligned}
    d_i d_j &= d_{j-1} d_i && \text{if } i < j, \\
    s_i s_j &= s_{j+1} s_i && \text{if } i \leq j, \\
    d_i s_j &=
    \begin{cases}
    s_{j-1} d_i & \text{if } i < j, \\
    \mathrm{id} & \text{if } i = j \text{ or } i = j+1, \\
    s_j d_{i-1} & \text{if } i > j+1.
    \end{cases}
    \end{aligned}
    \]

    Given a set $S = S_0 \sqcup S_1 \sqcup \dots$ and functions $d_i: S_n\to S_{n-1}$, $s_i: S_n \to S_{n+1}$ satisfying the simplicial identities there is a unique simplicial set which has the same face and identity maps. This fact gives a second way to define simplicial sets, these two definitions are used interchangeably in this thesis.
\end{definition}

\emph{Notation:} The notation $[n] = \{0 < 1 < \dots < n\}$ is used frequently in this thesis, most importantly for long winded combinatorical proofs to prevent the constant use of temporary variables and unreadable subscripts. 

\newpage
\section{Definitions}
\subsection{Ranked Hypergraphs}
\begin{definition}
    A directed \emph{hypergraph} $(V,H)$ is a finite set of vertices $V$ and a set of \emph{hyperarcs} $H\subseteq \P(V)^2$.
\end{definition}

\emph{Notation:} A directed hypergraph containing no cycles is a Directed Acyclic Hypergraph (DAH).\\

\begin{definition}
A \emph{ranked term hypergraph} $(g, r, o, \L, A)$ consists of:
\begin{itemize}
\item A DAH $g$,
\item Sequences $r = (r_i)_{i\in[|r|]}$, $o = (o_i)_{i\in[|o|]}$ $r_i, o_i\in \P(V)$ denoting the root and variable interfaces. $o_i$ contains only maximal vertices. We refer to $(|r|, |o|)$ as the \emph{rank} of this graph.
\item An action set $A$ and a hyperarc labelling function $\L: H\to A$
\end{itemize}
\end{definition}
\emph{Notation:} In this thesis we refer to ranked term hypergraphs as just hypergraphs as we will only be working with this kind. $HG(n,m)$ is the set of ranked term  hypergraphs of rank $(n,m)$

A ranked hypergraph with $|r| = |o|$ is called symmetric.\\

\begin{example}
    In figure \ref{fig:DAH-example} a ranked hypergraph is drawn. Left is the root interface, of rank 3, on the right is the output interface of rank 2. 

    The full definition of this graph is as follows
    \begin{itemize}
        \item $V = \{1,2,3,4,5,6\}$
        \item $H = \{(\{1\}, \{4\}), (\{2\}, \{5,6\})\}$
        \item $r = (\{1\}, \{2\}, \{3\})$, $o = (\{4\}, \{3,6\})$
        \item $A = \{a,b\}$, $\L((\{1\}, \{4\})) = b$, $\L((\{2\}, \{5,6\})) = a$
    \end{itemize}
    
    \begin{figure}[h]
    \centering
    \begin{tikzpicture}
		\node [style={gra_point}, label={above:1}] (0) at (-1, 1.5) {};
		\node [style={gra_point}, label={above:2}] (1) at (-1, 0) {};
		\node [style={gra_point}, label={above:4}] (3) at (1, 1.5) {};
		\node [style={int_point}] (5) at (-3, 1.5) {};
		\node [style={int_point}] (6) at (-3, 0) {};
		\node [style={int_point}] (7) at (-3, -1.75) {};
		\node [style={int_point}] (8) at (3.75, 1.25) {};
		\node [style={int_point}] (9) at (3.75, -1.25) {};
		\node [style={gra_point}, label={above:6}] (10) at (1, -0.75) {};
		\node [style={gra_point}, label={above:5}] (11) at (1, 0.5) {};
		\node [label={above:a}] (12) at (0, 0) {};
		\node [label={above:b}] (13) at (0, 1.5) {};
		\node [style={gra_point}, label={above:3}] (14) at (0, -1.75) {};
		\draw [style={int_edge}] (5) to (0);
		\draw (1) to (12.center);
		\draw [style={int_edge}] (7) to (14);
		\draw [style={int_edge}] (6) to (1);
		\draw [style={int_edge}] (9) to (14);
		\draw [style={int_edge}] (9) to (10);
		\draw [style={int_edge}] (8) to (3);
		\draw [style={gra_edge}] (0) to (3);
		\draw [style={gra_edge}] (12.center) to (11);
		\draw [style={gra_edge}] (12.center) to (10);
    \end{tikzpicture}
    \caption{A Directed Acyclic Hypergraph}
    \label{fig:DAH-example}
\end{figure}
\end{example}


\subsection{Composition of ranked hypergraphs}
\begin{definition}
Let $G, F$ be hypergraphs such that $|o^G| = |r^F|$, their composition is defined as follows:
\begin{align}
    G\otimes F = (g', r', o^F, \L^G\sqcup \L^F, A^G\cup A^F)
\end{align}

We obtain $g' = (V,H)$ by the following procedure:

Define 
% \begin{enumerate}
%     \item $g' = (V,H) \defeq g^G \sqcup g^F$
%     \item For each $o^G_i\in o^G$, $v\in o^G_i$, if v is minimal and/or $r^F_i\neq\varnothing$:\\ $V\defeq V\setminus \{v\}$ and for each $(U, U')\in H$ with $v\in U'$, set $U' \defeq (U' \cup r^F_i) \setminus \{v\}$\\
% \end{enumerate}

\begin{align*}
    V = (V^G + V^F) \setminus \bigcup_{i\in [|o^G|]}o^G_i
\end{align*}

To get the hyperarcs we keep all elements but replace a vertex if it exists in an output, to do this neatly we define a pair of functions:

\[
\psi_{r^F, o^G}(v) := 
\begin{cases}
\displaystyle\bigcup_{\substack{i \in [n] \\ v \in o^G_i}} r^F_i & \text{if } \exists i \in [n] \text{ such that } v \in o^G_i \\
\{v\} & \text{otherwise}
\end{cases}
\]

\[
\Psi_{r^F, o^G}(U') := \bigcup_{v \in U'} \psi_{r^F, o^G}(v)
\]

\[
H := \left\{ \left(U, \Psi_{r^F, o^G}(U')\right) : (U, U') \in H^G \right\} \cup H^F
\]

So for all $i$, in each arc $v$ that ends in a vertex in $o_i^G$, we replace that vertex in the arc with $r_i^F$. 

And we obtain $r'$ by taking over the original $r^G$ and `connecting through' for vertices which are both minimal and maximal:
\begin{align*}
    r'_i = \Psi_{r^F, o^G}(r^G_i)
\end{align*}
\end{definition}

This composition allows for an identity $id_n$ namely $id_n = (([n], \varnothing), (\{i\})_{i\in n}, (\{i\})_{i\in n})$.\vspace{5pt}

\begin{lemma}\label{lem:psieq}
    Let $G,K,F$ be a DAH and $U\subseteq V^G$ a set of vertices. This identity holds:
    \[
        \Psi_{r^F, o^K}\circ \Psi_{r^K, o^G}(U) = \Psi_{r^F\otimes r^K, o^G}(U)
    \]
\end{lemma}

\begin{proof}
    We prove the lemma for a singleton, this then extends to all subsets $U\subseteq V^G$. Let $v\in V^G$, we proceed by using the definitions:
    \begin{align*}
        \Psi_{r^F, o^K}\circ \psi_{r^K, o^G}(v) &= 
        \begin{cases}
            \{v\} & v\notin \cup_i o_i^G\\
            \Psi_{r^F, o^K}\left(\bigcup_{\substack{i \in [|r^K|] \\ v \in o^G_i}} r^K_i\right) & \text{else}
        \end{cases}\\
        \intertext{Because both unions are finite we can exchange them: }
        &= 
        \begin{cases}
            \{v\} & v\notin \cup_i o_i^G\\
            \bigcup_{\substack{i \in [|r^K|] \\ v \in o^G_i}} \Psi_{r^F, o^K} (r^K_i) & \text{else}\\
        \end{cases}\\
        \intertext{By definition $r^{K\otimes F}_i \defeq \Psi_{r^F, o^K} (r^K_i)$ and hence}
        &=
        \begin{cases}
            \{v\} & v\notin \cup_i o_i^G\\
            \bigcup_{\substack{i \in [|r^{K\otimes F}|] \\ v \in o^G_i}} r^{K\otimes F}_i & \text{else}\\
        \end{cases}\\
        &= \psi_{r^F\otimes r^K, o^G}(v)
    \end{align*}\\

    Taking unions of singletons then yields:
    \[
        \Psi_{r^F, o^K}\circ \Psi_{r^K, o^G}(U) = \Psi_{r^F\otimes r^K, o^G}(U)
    \]
\end{proof}

Which lets us prove the following theorem:

\begin{theorem}\label{thm:assoc}
The sequential composition of ranked hypergraphs has the following properties:
    \begin{enumerate}
        \item $\otimes$ is associative
        \item $id_n \otimes G = G = G \otimes id_m$ up to renaming of vertices
    \end{enumerate}
\end{theorem}

\begin{proof}
Let $G,K,F$ be ranked hypergraphs. It is clear from the definition that $(G \otimes K) \otimes F = G \otimes (K \otimes F)$ if and only if their graphs and root interfaces are equal.
     
Let $r$, $r'$ be the root interfaces for $(G \otimes K) \otimes F$, $G \otimes (K \otimes F)$ respectively.  
We expand the definition and use lemma \ref{lem:psieq} to show that $r = r'$. Let $i\in [|r^G|]$, we get:

\begin{align*}
    r_i = \Psi_{r^F, o^k}(r^{G\otimes K}_i) = \Psi_{r^F, o^k}(\Psi_{r^K, o^G}(r^G_i)) = \Psi_{r^{K\otimes F}, o^G}(r_i) = r'_i\\
\end{align*}


    Let $g = (V, H)$, $g' = (V', H')$ be the graphs for $(G \otimes K) \otimes F$, $G \otimes (K \otimes F)$ respectively.

    We expand the definition:
    \begin{gather*}
        V = (V^{G\otimes K} + V^F) \setminus \bigcup_{i\in [|o^K|]}o_i^K\\
        = (((V^G + V^K) \setminus \bigcup_{i\in[|o^G|]}o^G_i) + V^F) \setminus \bigcup_{i\in [|o^K|]}o_i^K
        \intertext{From disjointness the inclusion of $V^G$, $V^K$, and $V^F$ into their coproduct we get}
        = (V^G + ((V^K + V^F) \setminus \bigcup_{i\in[|o^K|]}o^K_i)) \setminus \bigcup_{i\in [|o^G|]}o_i^G = V'
    \end{gather*}

    Lastly for the arcs:
    \begin{align*}
        H &= \left\{ \left(U, \Psi_{r^F, o^K}(U')\right) : (U, U') \in H^{G\otimes K} \right\} \cup H^F\\
        &= \left\{ \left(U, \Psi_{r^F, o^K}(\Psi_{r^K, o^G}(U'))\right) : (U, U') \in H^G \right\}
        \cup \left\{ \left(U, \Psi_{r^F, o^K}(U')\right) : (U, U') \in H^K \right\} \cup H^F\\
        \intertext{By lemma \ref{lem:psieq} we have}
        &= \left\{ \left(U, \Psi_{r^{F\otimes K}, o^G}(U')\right) : (U, U') \in H^G \right\}
        \cup \left\{ \left(U, \Psi_{r^F, o^K}(U')\right) : (U, U') \in H^K \right\} \cup H^F\\
        &= \left\{ \left(U, \Psi_{r^{K \otimes F}, o^G}(U')\right) : (U, U') \in H^G \right\} \cup H^{K \otimes F}\\
        &= H'
    \end{align*}
\end{proof}

\subsection{Simplicial set over ranked term graphs}
\begin{definition}
    Let $V$ be the vertex set of a ranked hypergraph.\\
    We define the monoid $\M = (\P(V)^2, (\varnothing, \varnothing), \cup\times\cup)$. 
\end{definition}

From this monoid we define a simplicial set using the nerve construction.

\begin{definition}
    The \emph{nerve} $N(\M)$ of the monoid $\M$ is the simplicial set where:
    \begin{align*}
        N(\M)_n = \M^n\\
        d_i(m_1,\dots,m_n) = 
        \begin{cases}
            (m_1,\dots,m_i \cup\times\cup m_{i+1}, \dots, m_n)  0 < i < n\\
            (m_2,\dots, m_n)  i = 0\\
            (m_1,\dots,m_{n-1})  i = n
        \end{cases}\\
        s_i(m_1,\dots,m_n) = (m_1, \dots, m_i, (\varnothing, \varnothing), m_{i+1}, \dots, m_n)\\
    \end{align*}
\end{definition}

\begin{definition}
Define the simplicial set $\H$ by $\H_n = HG(n,n)$. The face and degeneracy maps of $\H$ are defined to be the unique maps $d^{\H}$, $s^{\H}$ making the following diagrams commute:

\[
\begin{tikzcd}
\H_n \arrow{r}{d_i^{\H}} \arrow{d}{\pi_n} & \H_{n-1} \arrow{d}{\pi_{n-1}} \\
N(\M)_n \arrow{r}{d_i^\M} & N(\M)_{n-1}
\end{tikzcd}
\quad
\begin{tikzcd}
\H_n \arrow{r}{s_j^{\H}} \arrow{d}{\pi_n} & \H_{n+1} \arrow{d}{\pi_{n+1}} \\
N(\M)_n \arrow{r}{s_j^\M} & N(\M)_{n+1}
\end{tikzcd}
\]

Where $\pi_n$ is the projection onto the interfaces given by: $\pi_n((g, r, o, \L, A)) = ((r_i, o_i))_{i\in [n]}$. That is, the face and degeneracy maps of $\H$ are defined by the underlying monoidal nerve on the interfaces.
\end{definition}

$\H$ is a simplicial set precisely because we inherit the face and degeneracy maps from $N(\M)$:\\

\begin{lemma}
    $\H$ is indeed a simplicial set.
\end{lemma}

\begin{proof}
    $\pi_n$ is a simplicial morphism by commutation of the given diagrams. Since $N(\M)$ is a simplicial set by definition and the diagrams commute the simplicial identities also hold for $d^{\H}$ and $s^{\H}$. Therefore $\H$ is a simplicial set.
\end{proof}

\subsection{Coalgebraic behaviour}
To add behaviour to the hypergraph simplicial set we define pointed F-coalgebras by the endofunctor

\begin{definition}
    \[
        F: sSet \to sSet, \quad FX = \B \times \finP(X \sqcup \uparrow X)^{\H}
    \]
    
    Here each state therefore outputs a boolean in $\B$ indicating if it's an accepting state, and transition maps $\H \to \finP(X \sqcup \uparrow X)$. Here the $\finP$ is the finite powerset on simplicial sets $\finP: sSet \to sSet$ applied to $X\sqcup \uparrow X$.
    
    Let $\Delta$ be the simplex category, we define $(-)^{\rhd}$ to be the functor which adds a new maximal element to a powerset. Precomposing this with the presheaf $X$ gives $\uparrow X = X\otimes (-)^{\rhd}$. Thus from this construction, given an $x\in X_n$ we can only transition to elements in $X_m$ where $m\geq n$.
    
    We also define $F(f) = \B\times\finP(f \sqcup \uparrow f)^{\H}$, which just applies $f$ to all transitioned-to elements.
\end{definition}
\vspace{2em}
\begin{definition}
    A pointed $F$-coalgebra over a functor $F$ is a triple $(X, \alpha: X\to FX, x_0)$ where $X$ is the \emph{carrier set} and $x_0$ is the \emph{base} or in our case an \emph{initial state}.
\end{definition}

Through this definition we find out what a coalgebra does on our set. If we follow the repeated iteration of this coalgebra
\[
    X\xrightarrow{\alpha} FX \xrightarrow{F\alpha} FFX \xrightarrow{FF\alpha} \dots
\]

We get by definition of $F(f)$ that $\alpha$ gets recursively applied to the transitioned-to elements. This will look like

\begin{align*}
    \alpha(x_0) &= (0, \{g_{01}\mapsto x_{01}, g_{02}\mapsto x_{02}, \dots\})\\
    \implies \alpha\circ\alpha(x_0) &= (0, \{(g_{01}\mapsto (0, \{g_{011}\mapsto x_{011}, g_{012}\mapsto x_{012}\})),\\
    &\hspace{40pt}(g_{02}\mapsto (0, \{g_{021}\mapsto x_{021}, g_{022}\mapsto x_{022}\})), \dots\})
\end{align*}

This is a tree structure with $x_0$ as root, $x_{01}, x_{02}, \dots$ as children etcetera. The transitions are through the hypergraphs. Non-accepting states are red, and accepting states are indicated by a green node. In figure \ref{ex:coalg-tree} a tree is visualized for some coalgebra. Here $x_{012}$ is an accepting state.

\begin{figure}[h]
    \centering
    \begin{tikzpicture}[->, level distance=3cm,
            level 1/.style={sibling distance=4cm},
            level 2/.style={sibling distance=3cm},
            edge from parent node/.style={black},
            every node/.style={red}]
    \node (Root) {$x_0$}
        child {
            node {$x_{01}$}
            child { 
                node {$x_{011}$}
                child {
                    node {$\iddots\hspace{20pt}$}
                }
                child {
                    node {$\hspace{20pt}\ddots$}
                }
                edge from parent node[left, black] {$g_{011}$} 
            }
            child { 
                node [Green] {$x_{012}$}
                child {
                    node {$\vdots$}
                }
                edge from parent node[right, black] {$g_{012}$}
            }
            edge from parent node[left, black] {$g_{01}$}
        }
        child {
            node {$x_{02}$}
            child { 
                node {$x_{021}$}
                child {
                    node {$\vdots$}
                }
                edge from parent node[left, black] {$g_{021}$}
            }
            child { 
                node {$x_{022}$}
                child {
                    node {$\iddots\hspace{20pt}$}
                }
                child {
                    node {$\hspace{20pt}\ddots$}
                }
                edge from parent node[right, black] {$g_{022}$}
            }
            edge from parent node[right, black] {$g_{02}$}
        };
    \end{tikzpicture}
    \caption{Tree from iteration of some $F$-coalgebra}
    \label{ex:coalg-tree}
\end{figure}

\begin{definition}
    Given coalgebras $(X, \alpha: X\to FX, x_0)$, $(Y, \beta: Y\to FY, y_0)$, a function $f: X\to Y$ is a homomorphism if and only if
    \begin{align*}
        F(f)\otimes \alpha = \beta \otimes f\\
        f(x_0) = y_0
    \end{align*}
\end{definition}

While a strict characterisation of the homomorphisms has evaded me thus far. It is known that the following property holds:

\begin{lemma}
    The homomorphisms on $FX = \H \times \finP(X \sqcup \uparrow X)$ preserve transitions. That is, if $\alpha (x) = (g_x, \{x_1, x_2, \dots\})$ then $\beta(f(x)) = (g_x, \{f(x_1), f(x_2), \dots\})$
\end{lemma}

\newpage
\section{Related Work}\label{relatedwork}

\section{Conclusions and Further Research}\label{conclusions}

\bibliographystyle{alpha}
\bibliography{bibliography}
\addcontentsline{toc}{section}{References}


%\appendix
%appendices here --- if any

\end{document}